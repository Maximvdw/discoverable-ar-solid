\section{Introduction and Related Work}
In Augmented reality~(AR), virtual objects are superimposed onto the physical world. Since its early days, AR has significantly advanced to portable devices such as smartphones and head-mounted displays~\cite{10.5555/2427126}, enabling its use in regular environments such as office buildings to superimpose virtual information to the physical world.

Superimposed virtual objects or information can be anchored to physical objects and walls within these physical environments~\cite{10.1145/3301275.3302278,kalaitzakis2021fiducial}. These virtual objects can range from textual information such as timers or instructions to images, videos or interactive elements~\cite{6948506}. However, there is a lack of AR solutions that enable collaboration on the creation of these AR environments without the need for proprietary applications.

A conceptual content sharing solution for AR, offering a peer-to-peer and client-server collaboration solution using events that indicate changes to the virtual objects, was proposed in~\cite{236306}. While the solution proposed in~\cite{236306} supports collaboration in AR, it still requires a complex and proprietary peer-to-peer communication framework and does not tackle the problem of different devices using a different frame of reference~\cite{mou2004frames}. 

We propose an interoperable solution that enables users to create their personal AR environments that can be discovered in the physical world by other users using AR-enabled devices. Unlike other existing work such as XSpace~\cite{10.1145/3567721}, we aim for an synthetic and semantic interoperable solution that decentralises the collaborative aspect of AR environments, while also enabling the sharing of crucial environmental information such as common reference spaces. Our interoperable solution should enable the unambiguous description of the virtual environment in a way that can be accessed and understood by other applications. In this paper we focus on the syntactic- and processing interoperability~\cite{doi:10.1080/13614570109516975,khan2013process} of the virtual environments. The semantic interoperability of 3D environments through common vocabularies is limited to illustrate the basic concept.
